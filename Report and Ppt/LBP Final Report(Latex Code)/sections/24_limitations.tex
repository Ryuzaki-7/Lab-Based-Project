\vspace{1cm}
\section{{{\fontsize{17}{21}\selectfont \textbf{Limiations}}}}
\setlength{\columnsep}{1.5cm}

\begin{multicols}{2}
While our model demonstrated impressive results in detecting objects in camouflaged environments, there are still several limitations to our approach.\\
One major limitation is that the model heavily relies on the quality and diversity of the training data. If the training dataset is limited or biased, the model may not generalize well to new and unseen scenarios. Moreover, our approach of combining spectral information through PCA may not be optimal for all scenarios, and more advanced fusion techniques may be required.\\
Another limitation is that our model may struggle with detecting camouflaged objects that closely resemble the background, such as objects with similar color or texture patterns. Additionally, our model may not be able to detect objects that are partially occluded or partially obscured by other objects.\\
Lastly, our approach of using a single model may not be suitable for all types of camouflaged object detection scenarios. For example, detecting small and fast-moving objects may require a different type of model with higher temporal resolution.\\
Overall, these limitations indicate that there is still much room for improvement in the field of camouflaged object detection, and further research is required to address these challenges.
\end{multicols}

\vspace{0.5cm}
{\color{gray}\hrule}
\vspace{0.5cm}