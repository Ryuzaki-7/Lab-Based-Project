\section{{{\fontsize{17}{21}\selectfont \textbf{Research Gaps}}}}
\setlength{\columnsep}{1.5cm}

\begin{multicols}{2}

Despite the promising results of the previous studies on SINet-based camouflage detection, there are still several research gaps that need to be addressed in order to improve the performance of the models in detecting camouflaged objects.\\
One of the main research gaps is the limited use of multispectral imaging for training deep learning models. While the use of RGB images has been widely explored in the literature, only a few studies have explored the use of multispectral imaging for training deep learning models for camouflage detection. This is a significant gap since multispectral imaging provides additional spectral information that can improve the detection of camouflaged objects.\\
Another research gap is the lack of research on the optimal number of spectral bands required for detecting camouflaged objects. While some studies have used multispectral images with five spectral bands, others have used images with fewer spectral bands. There is a need for further research to determine the optimal number of spectral bands required for detecting camouflaged objects, as this can impact the performance of the models.\\
In addition, there is a need for further research on the impact of different feature extraction techniques on the performance of SINet-based models for camouflage detection. While some studies have used PCA for feature extraction, others have used other techniques such as wavelet transforms and convolutional neural networks. There is a need for further research to determine the most effective feature extraction technique for detecting camouflaged objects in complex environments.\\
Finally, there is a need for further research on the robustness of SINet-based models to different types of camouflage. While some studies have focused on the detection of natural camouflage, such as vegetation, others have explored the detection of artificial camouflage, such as urban camouflage. There is a need for further research to determine the robustness of the models to different types of camouflage, as this can impact their effectiveness in real-world scenarios.\\
Addressing these research gaps can help to improve the performance of SINet-based models for detecting camouflaged objects and inform the development of more effective solutions for military and security applications.
\end{multicols}

\vspace{0.5cm}
{\color{gray}\hrule}
\vspace{0.5cm}