\vspace{1cm}
\section{{{\fontsize{17}{21}\selectfont \textbf{Conclusion}}}}
\setlength{\columnsep}{1.5cm}

\begin{multicols}{2}
In our study, we conducted an extensive investigation into object detection in camouflaged environments. We utilized both RGB and multi-spectral datasets to evaluate the effectiveness of different object detection techniques. In addition, we created our own dataset using drone imagery to simulate real-world scenarios and enhance the accuracy of our findings.\\
To combine the data from different spectral bands, we performed data fusion by aligning the images and using PCA to extract a 3-channelled image. This allowed us to combine the different spectral information into a single image, making it easier to detect objects in camouflaged environments.\\
To accomplish these tasks, we utilized SINet as our model, which has been demonstrated to be highly competitive with other state-of-the-art technologies in this field. We evaluated the performance of our model on both the RGB and multi-spectral datasets and found that it demonstrated excellent camouflage detection capabilities.\\
However, despite the impressive results, our model still has certain limitations. This is because the field of object detection in camouflaged environments is still relatively new in the deep learning domain. As such, there is still much room for improvement, and we believe that future research and advancements will continue to enhance the effectiveness of object detection in camouflaged environments.\\
Overall, our study provides valuable insights into the challenges of object detection in camouflaged environments and the potential for deep learning techniques to overcome these challenges. The techniques we employed in our study can serve as a foundation for future research in this field, with the ultimate goal of improving object detection in real-world applications.
\end{multicols}

\vspace{0.5cm}
{\color{gray}\hrule}
\vspace{0.5cm}