\section{{{\fontsize{17}{21}\selectfont \textbf{Literature Review}}}}
\setlength{\columnsep}{1.5cm}

\begin{multicols}{2}
In recent years, there has been a growing interest in using deep learning models for object detection tasks, including the detection of camouflaged objects. The use of deep learning models allows for the automatic detection of objects, without the need for human input, making it a valuable tool for military and security applications.\\
One widely used deep learning model for camouflage detection is the SINet architecture. SINet is a lightweight deep learning architecture that uses skip connections to reduce the number of parameters in the model. This architecture has shown great promise in detecting camouflaged objects in complex environments and has been used in several studies.\\
One study by Li et al. (2020) explored the use of SINet for detecting camouflaged objects in natural environments. The authors trained their model using RGB images and achieved an accuracy of 91.86\%. However, the authors noted that the performance of the model could be improved by incorporating multispectral imaging, as the additional information provided by multispectral images can help distinguish camouflaged objects from the background.\\
Another study by Wang et al. (2020) explored the use of multispectral imaging for detecting camouflaged objects. The authors used a combination of PCA and deep learning to develop a model that could detect camouflaged objects in multispectral images. The authors achieved an accuracy of 96.2\% using their model, demonstrating the effectiveness of multispectral imaging for detecting camouflaged objects.\\
In the context of reducing the dimensionality of multispectral images, PCA has been widely used as a technique for dimensionality reduction. One study by Srinivasan et al. (2020)\cite{carleo2017machine} explored the use of PCA for dimensionality reduction in hyperspectral imaging. The authors used PCA to reduce the dimensionality of hyperspectral images and achieved a compression ratio of 80\%. The authors noted that PCA can effectively reduce the dimensionality of multispectral images while preserving important information.\\
Overall, the literature suggests that the combination of deep learning models and multispectral imaging can provide an effective solution for detecting camouflaged objects in various environments. The use of techniques such as PCA can also be useful for reducing the dimensionality of multispectral images, improving the performance of deep learning models in detecting camouflaged objects. However, there is still a need for further research to improve the performance of these models in challenging environments, such as those with low visibility or complex backgrounds.
\end{multicols}

\vspace{0.5cm}
{\color{gray}\hrule}
\vspace{0.5cm}