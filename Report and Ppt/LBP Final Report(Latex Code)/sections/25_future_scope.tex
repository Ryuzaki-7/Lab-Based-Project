\vspace{1cm}
\section{{{\fontsize{17}{21}\selectfont \textbf{Future Scope}}}}
\setlength{\columnsep}{1.5cm}

\begin{multicols}{2}
The field of camouflaged object detection has great potential for future advancements, and there are several areas that could be explored to improve the effectiveness of detection models. Few of them are listed below: \\

\textit{i)} Concealed object detection under limited conditions: few/zero-shot learning, weakly supervised learning, unsupervised learning, self-supervised learning, limited training data, unseen object class, etc.\\

\textit{ii)} Camouflaged Object Detection combined with other
modalities: Text, Audio, Video\cite{ma2022structured}, RGB-D, RGB-T, 3D,etc.\\

\textit{iii)} Existing deep-based methods extract the features in a fully supervised manner from images annotated with object-level labels. However, the pixel-level annotations are usually manually marked by LabelMe or Adobe Photoshop tools with intensive professional interaction. Thus, it is essential to utilize weakly/semi (partially) annotated data for training in order to avoid heavy annotation costs.\\

\textit{iv)} Existing concealed data is only based on static images or dynamic videos. However, concealed object detection in other modalities can be closely related in domains such as pest monitoring in the dark night, robotics, and artist design.
\end{multicols}

\vspace{0.5cm}
{\color{gray}\hrule}
\vspace{0.5cm}